\documentclass[11pt]{article}
\usepackage{hyperref}
\begin{document}

\begin{center}
    Teaching Statement\\
    Kerry Back\\ January 2024
\end{center}
\vskip\baselineskip

I have taught every topic in a finance curriculum at some point in my career.  Recently, I have been focused on helping students to expand their financial analysis toolkits beyond spreadsheets.  Some familiarity with programming, and in particular python, is something that I think our students will need in the future (see, for example, Microsoft's beta testing of Python in Excel) and should be learning now.  I teach students in a variety of programs: Ph.D. students in business and economics, MBA students, undergraduate business students, and students in the Masters of Data Science (MDS) program in the computer science department.  Outside of teaching asset pricing theory to Ph.D. students in BUSI 521 (using my book \textit{Asset Pricing and Portfolio Choice Theory}, Oxford University Press, 2nd ed., 2016), all of my other teaching includes python to some extent.  This has been most successful with Ph.D. students (BUSI 520) and in the MDS courses (BUSI 721 and 722).  It has been less successful with MBA students.  The MBA students get valuable exposure (and exposure that they largely seem to value) but not enough exposure to actually acquire skills, given that almost all other courses in the business school are still spreadsheet-based.  I am optimistic that generative AI will help to bridge this gap.  I am pioneering a course in the second half of the spring semester for MBA students (MGMT 675: AI-Assisted Financial Analysis) that is aimed at developing prompt engineering skills to get ChatGPT to write python code for financial analyses.  I will cover some applications the students have seen before but with different methods and also some applications that are infeasible with spreadsheets.

Another major teaching-related emphasis of mine in the past few years has been the development of a website that illustrates investment concepts with the help of interactive figures and tables (\href{https://learn-investments.rice-business.org}{https://learn-investments.rice-business.org}).  I developed the website jointly with Kevin Crotty, an Associate Professor of Finance at JGSB.  The website contains over 80 pages, each of which illustrates a concept in investments.  Many of the pages pull data from online sources, and they all allow user control of inputs -- for example, selecting a stock ticker or entering cash flows.  Kevin and I won the Financial Management Association Innovation in Teaching Award for 2023 for developing the website and making it available.  Python is not needed to use the website, but we do provide python code to conduct the analyses for students who want to learn it.

My contributions in teaching led to my being selectd recently by the Financial Management Association to be one of two nominees for the position of Vice President for Education.  The election will be in the spring.

\end{document}