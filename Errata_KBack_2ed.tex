\documentclass[11pt]{book}
\RequirePackage{natbib,amsmath,amsthm,array,graphicx,footmisc,amsfonts,geometry,fancyvrb,chngcntr,minitoc}
\setlength{\parindent}{5ex}

\newcommand{\hiddensection}[1]{
	\stepcounter{section}
	\section*{\arabic{chapter}.\arabic{section}\hspace{1em}{#1}}
}

	
\counterwithin{table}{chapter}

\newenvironment{mypetit}{\centerline{\rule[0.2\baselineskip]{1in}{0.15mm}}\noindent\small}{}
\newenvironment{mypetitenum}{\centerline{\rule[0.2\baselineskip]{1in}{0.15mm}}\noindent\small}{}

\def\next{\vskip \baselineskip\noindent}
\newcommand{\mybox}[1]{\next\fbox{\parbox{4.55in}{#1}}\next}
\newcommand{\listtype[1]}{\renewcommand{\labelenumi}{#1}}
\newcommand{\vc}{^{\text{vec}}}
\newcommand{\mv}{_{\text{m}}}
\newcommand{\zb}{_{\text{z}}}
\newcommand{\cm}{_{\text{c}}}
\newcommand{\vct}{^{\text{vec}}}
\newcommand{\wt}[1]{\widetilde{#1}}
\newcommand{\tm}{\ensuremath{ t\!-\!1} }
\newcommand{\Tm}{\ensuremath{ T\!-\!1} }
\newcommand{\pr}{\ensuremath{\mathbb{P}}}
\newcommand{\qr}{\ensuremath{\mathbb{Q}}}
\newcommand{\mye}{\ensuremath{\mathsf{E}}}
\newcommand{\sr}{\ensuremath{\mathbb{S}}}
\newcommand{\fr}{\ensuremath{\mathbb{F}}}
\newcommand{\price}{\ensuremath{\mathcal{P}}}
\newcommand{\myreal}{\ensuremath{\mathbb{R}}}
\newcommand{\excise}[1]{\vskip 0.5\baselineskip \textit{#1} \vskip 0.5\baselineskip}
\newcommand{\CE}{\xi}
\newcommand{\sectspace}{\;\;\;\;}
\newcommand{\D}{\mathrm{d}}
\newcommand{\E}{\mathrm{e}}
\newcommand{\eqdef}{\;\buildrel \text{d{}ef}\over =\;}
\newcommand{\eqquest}{\;\buildrel \text{?}\over =\;}
\newcommand{\halfskip}{\vskip 0.5\baselineskip\noindent}
\newcommand{\onevector}{\iota}
\newcommand{\Rvector}{\mathbf{R}}
\newcommand{\sol}{\textbf{Solution:} \hspace{2ex}}
\newcommand{\be}{\begin{enumerate}\renewcommand{\labelenumi}{(\alph{enumi})}}
\newcommand{\ee}{\end{enumerate}}
\newcommand{\bq}{\begin{equation}}
\newcommand{\eq}{\end{equation}}

\theoremstyle{definition}
\newtheorem{prob}{}[chapter]

\newcommand{\teff}{\tau_{\text{eff}}}

\DeclareMathOperator{\var}{var} \DeclareMathOperator{\stdev}{stdev}
\DeclareMathOperator{\cov}{cov} \DeclareMathOperator{\corr}{corr}
\DeclareMathOperator{\M}{M} \DeclareMathOperator{\N}{N}
\DeclareMathOperator{\nd}{n} \DeclareMathOperator{\Cov}{Cov}
\DeclareMathOperator{\Prob}{prob} \DeclareMathOperator{\Var}{Var}
\DeclareMathOperator{\argmax}{argmax}
\DeclareMathOperator{\proj}{proj}
\DeclareMathOperator{\sign}{sign}

\newcommand{\ptext}[1]{\Prob(\text{#1})}

\geometry{verbose,letterpaper,tmargin=1in,bmargin=1.25in,lmargin=1in,rmargin=1in,headheight=0.2in,footskip=0.5in}
\renewcommand{\baselinestretch}{2}
\setlength{\headsep}{2\baselineskip}
\setlength{\footnotesep}{\baselineskip}

\newcommand{\bi}{\begin{itemize}}
\newcommand{\ei}{\end{itemize}}
\newcommand{\im}{\item}

\newcommand{\notes}[1]{
\addtocontents{toc}{\setcounter{tocdepth}{0}}
\addtocontents{minitoc}{\setcounter{tocdepth}{0}}
\section{\sectspace Notes and References}\label{#1}
\addtocontents{toc}{\protect\setcounter{tocdepth}{2}}
\addtocontents{minitoc}{\protect\setcounter{tocdepth}{2}}
}


\setlength{\headsep}{2\baselineskip}

\begin{document}
\VerbatimFootnotes
%%%%%%%%%%%%%%%%%%%%%%%%%%%%%%%%%%%%%%%%%%%%%%%%%%%%%%%%%%%%%


%%%%%%%%%%%%%%%%%%%%%%%%%%%%%%%%%%%%%%%%%%%%%%%%%%%%%%%%%%%%%

\author{Kerry Back}
\title{Errata\\ Asset Pricing and Portfolio Choice Theory\\
Second Edition\\
}
\date{\today}
\maketitle


%%%%%%%%%%%%%%%%%%%%%%%%%%%%%%%%%%%%%%%%%%%%%%%%%%%%%%%%%%%%%

\pagenumbering{arabic}
\begin{enumerate}
\item The utility function $u$ is missing from the left-hand side of (1.3).  It should read 
$$u(w-\pi) = \mye[u(w+\tilde{\varepsilon})]\,.$$
\item Page 37, line 10: $\var(\tilde{R}_n)$ should be $\var(\tilde{R}_2)$.
\item Exercise 2.7(b) needs the additional assumption that $c_1>c_0$.  Also, there shouldn't be a tilde on $c_1$, because it is not random.
\item Part (e) of Exercise 8.1 relies on (9.6) and should not be assigned until Chapter 9 is covered.
\item Exercise 9.1 requires the assumption that the investor has log utility (this should be evident anyway).
\item Exercise 10.1 requires a link between $\nu$ and $\delta$ that is not specified in the problem.  The displayed formula before (10.32) on p. 251 should be used.
\item Lognormal consumption growth should be assumed in Exercise 11.1.
\item Here is some advice regarding Exercise 13.2: The exercise is simpler if it is assumed that $B$ is the only Brownian motion in the economy.  In this case (13.58) is equivalent to
	$$\frac{\D M}{M} = - r\,\D t - \lambda\,\D B\,.$$
	If there are other Brownian motions, then some regularity condition is needed to ensure a local martingale is a martingale.  See Exercise 15.2 for such a result.
	\item A term is missing from the right-hand side of (14.22).  It comes from the $-cJ_w$ term in (14.19).  We are writing $c=f(t,w)$, so equation (14.22) should read
	$$0 = u(f(t,w)) - \delta J + J_t + wJ_w \left(r - \frac{f(t,w)}{w} - \frac{\kappa^2}{2}\frac{J_w}{wJ_{ww}}\right)\,.$$
\item In Exercise 14.2, $\eta_{hj}$ should be replaced by $\eta_{hj}/X_j$ and $\eta_j$ should be replaced by $\eta_j/X_j$ to be consistent with the definitions in the chapter (which changed between the 1st and 2nd editions of the book). So, the equation that is given in the exercise should be
$$W_h \pi_h = \tau_h\Sigma^{-1}(\mu-r\iota) - \sum_{j=1}^\ell \tau_h\frac{\eta_{hj}}{X_j}\Sigma^{-1}\sigma\nu_j\,.$$
Equation (14.33), which is to be derived in part (a), should be
$$\mu-r\iota = \alpha W \Sigma \pi + \sum_{j=1}^\ell \frac{\eta_j}{X_j}\sigma\nu_j\,.$$
\item The $rf$ on the right-hand side of the PDE (15.3) should not be present (it was copied from (15.1) but here it is subsumed in the $-mf_mr$ term on the left-hand side).  The correct PDE is
\begin{multline*}
g+ f_t  + \sum_{i=1}^\ell f_{x_i}(\phi_i-\nu_i'\lambda) - mf_m(r-\lambda'\lambda) \\ + \frac{1}{2}\sum_{i=1}^\ell \sum_{j=1}^\ell f_{x_ix_j} \nu_i'\nu_j + \frac{1}{2}m^2f_{mm}\lambda'\lambda - m \sum_{i=1}^\ell f_{mx_i}\nu_i'\lambda
= 0\,.
\end{multline*}
\item In Exercise 15.4, the same symbol $\rho$ is used for the correlation between the two stochastic processes and also for relative risk aversion.  This can be fixed by replacing the equation for $\D \mu$ by the following (which uses $\eta$ for the correlation): 
	$$\D \mu  = \kappa(\theta-\mu)\,\D t + \gamma\left[\eta\,\D B_1 + \sqrt{1-\eta^2}\,\D B_2\right]\,.$$
	\item In Exercise 18.4, the one instance of $S(r,Y)$ should be $S(r,Z)$.
\item In the example on page 505, the decimal point is in the wrong place for the optimal coupons.  In the static model, the optimal coupon is 2.69\% of $X_0$.  In the dynamic model, the optimal coupon is 1.58\% of $X_0$.
\item The second $\gamma$ in the displayed expression in Exercise 19.4(c) should be $\beta$.  That is, the expression should be
$$\frac{c}{r} + as^{-\gamma} + bs^\beta\,.$$
\item On page 541, regading (20.35), the ``volatility of $X$'' should be interpreted as the instantaneous standard deviation of $\D X$, not the instantaneous standard deviation of $\D X/X$.  Likewise, replace ``risk of a stock return'' on page~541 and in Exercise 20.3 with ``risk of a stock price change.''
\item In Exercise 20.2, replace the sentence ``Using the formula for the constant $c_\pi$ derived in the previous exercise, specify a condition on the parameters~$A$,~$\alpha$,~$w$, $\mu_z$, and $\sigma_z$ that is equivalent to $\delta>0$''  with ``Specify a condition on the parameters $r$, $\varepsilon$, $\alpha$, $\mu_z$, and $\sigma_z$ that is equivalent to $\delta>0$.''
\item Exercise 23.4(a) should ask for the equation for $\D \hat{\mu}$ instead of $\D \mu$.
\item There is a mistake in Exercise~24.1(b). The inventory control effect creates positive correlation that exactly offsets the negative correlation created by bid-ask bounce in this example.  So, the problem should say ``show that the serial correlation of the transaction price changes is zero.''  
\end{enumerate}


\end{document}
