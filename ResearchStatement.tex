\documentclass[11pt]{article}
\begin{document}

\begin{center}
    Research Statement\\
    Kerry Back\\ January 2024
\end{center}
\vskip\baselineskip

Most of my research has been about the organization of markets and the effect of information asymmetry on market outcomes.  My 1992 paper, "Insider Trading in Continuous Time," (paper \#9 in the list below) made an important methodological contribution regarding the solution of a certain type of asymmetric information model.  That model is now often called the Kyle-Back model.  I extended and applied the model to various issues (see papers 13, 15, 17, 24, 25, 27, and 32).  Paper 13, "Asymmetric Information and Options," won the Best Paper Award at the Review of Financial Studies in 1993.  A precursor of \#27, "Activism, Strategic Trading, and Liquidity," won the Charles River Associates Award for the best paper in corporate finance at the Western Finance Association in 2014.   I have also developed and applied other models to study market organization (papers 20, 22, 29, 30, and 34). 

A second principal area of research has been on the theory of asset prices and tests of the theory (7, 8 16, 26, 28, and 31).  My early work in that area is methodological in nature.  The later work is more empirical.  Paper 26 is about mutual fund returns and shows that good performance on conventional criteria (alphas) is associated with bad performance on another important criterion (co-skewness) and vice versa.  Paper 31 shows that a recently developed statistic for predicting the market return does not perform well out of sample and that a natural adjustment to that statistic cannot be expected to perform well until we have much more data (at least a century's worth) to estimate it.  

My most highly cited paper (albeit by a narrow margin) is in neither of the areas described above.  Paper 18 is on auction theory and in particular on the inapplicability of the theory that existed at the time it was written to a proposal by the U.S. Treasury to modify the way the government borrows money.  It was reprinted in a book of classic papers on auction theory.  Early in my career, I worked on a variety of topics in mathematical programming (1, 4, and 5), mathematical economics (2, 3, 6, and 11), and econometric theory (10 and 12).  More recently, I have also worked on the game theory of corporate competition in investments (21) and disclosures (33).  

Economic theory helps us to think about the world.  It illuminates incentives and the likely consequences of those incentives.  I have tried to provide such illumination on various issues in securities markets, auctions, and corporate competition.  However, finance is largely an empirical discipline.  Increases in computational power have expanded the types of empirical analyses that are possible.  I have been engaged recently in exploring some of these (in particular, machine learning and sampling methods for Bayesian analysis).  I am excited at the prospect of applying those tools to better understand the properties of asset returns in the future while continuing to explore incentives in markets and elsewhere.

My research has been recognized by the profession in various ways, including editorships.  The most recent recognition is an invitation to give the Distinguished Speaker address at the Western Finance Association conference in June, 2024.  This is one of the top two finance conferences in the world each year.

\vskip 2\baselineskip

\noindent \underline{Refereed Publications and Working Papers}
\vskip\baselineskip

\begin{enumerate}
\item  Back, K., 1986, ``Continuity of the Fenchel Transform of Convex Functions,''
{\em Proceedings of the American Mathematical Society\/} {\bf 97},
661--667.
\item  Back, K., 1986,
``Concepts of Similarity for Utility Functions,'' {\em
Journal of Mathematical Economics\/} {\bf 15}, 129--142.
\item  Back, K., 1987, ``A Compact Space of Transitive Locally Non-Satiated
Preference Relations,'' {\em Economics Letters\/} {\bf 24}, 1987,
253--256.
\item  Back, K., and S. R. Pliska, 1987, ``The Shadow Price of Information in Continuous Time
Decision
Problems,'' {\em Stochastics\/} {\bf 22}, 151--186.
\item  Back, K., 1988, ``Convergence of Lagrange Multipliers and Dual Variables for
Convex Optimization Problems,'' {\em Mathematics of Operations
Research\/} {\bf 13}, 74--79.
\item  Back, K., 1988, ``Structure of Consumption Sets and Existence of Equilibria
in Infinite Dimensional Spaces,'' {\em Journal of Mathematical
Economics\/} {\bf 17}, 39--49.
\item  Back, K., and S. R. Pliska, 1991, ``On the Fundamental Theorem of Asset Pricing with an
Infinite
State Space,''
{\em Journal of Mathematical Economics\/} {\bf 20}, 1--18.
\item  Back, K., 1991, ``Asset Pricing for General Processes,'' {\em Journal of
Mathematical Economics\/} {\bf 20}, 371--395.
\item  Back, K., 1992, ``Insider Trading in Continuous Time,'' {\em Review of Financial
Studies\/} {\bf 5}, 387--409.
\item  Back, K., and D. P. Brown, 1992, ``GMM, Maximum Likelihood, and Nonparametric
Efficiency,'' {\em Economics Letters\/}
{\bf 39}, 23--28.
\item Back, K., 1993, ``Incomplete Markets and Individual Risks,''
{\em Economic Theory\/} {\bf 3}, 35--42.
\item  Back, K., and D. P. Brown, 1993, ``Implied Probabilities in GMM Estimators, ''
{\em Econometrica\/} {\bf 61}, 971--975.
\item Back, K., 1993, ``Asymmetric Information and Options,'' {\em Review of Financial
Studies} {\bf 6}, 435--472 (received 1993 best paper award from RFS).
\item Back, K., and J. F. Zender, 1993, ``Auctions of Divisible Goods: On the Rationale for the Treasury
Experiment,'' {\em Review of Financial Studies\/}
{\bf 6}, 733--764 (reprinted in Klemperer, Paul, ed., {\em The Economic Theory of Auctions}, 2000, Edward Elgar).
\item Back, K., and H. Pedersen, 1998, ``Long-Lived Information and Intraday Patterns,'' {\em Journal of Financial Markets\/}
{\bf 1}, 385--402.
\item Dybvig, P. H., Rogers, L. C. G., and K. Back, 1999,
``Portfolio Turnpikes,'' {\em Review of Financial Studies\/} {\bf 12}, 165--195.
\item Back, K., Cao, H., and G. Willard, 2000, ``Imperfect Competition
among Informed Traders,'' {\em Journal of Finance\/} {\bf 55}, 2117--2155 (nominated for Smith-Breeden award).
\item Back, K., and J. Zender, 2001, ``Auctions of Divisible Goods with Endogenous Supply,'' {\em Economics Letters\/} {\bf 73}, 29-34.
\item Back, K., and S. Baruch, 2004, ``Information in Securities Markets: Kyle Meets Glosten and Milgrom,'' \textit{Econometrica} \textbf{72}, 433--465.
\item Back, K., and S. Baruch, 2007, ``Working Orders in Limit Order Markets and Floor Exchanges,'' \textit{Journal of Finance} \textbf{61}, 1589--1621.
\item Back, K., and D. Paulsen, 2009, ``Open Loop Equilibria and Perfect Competition in Option Exercise Games,'' \textit{Review of Financial Studies} \textbf{22}, 4531--4552.
\item Back, K., and S. Baruch, 2013, ``Strategic Liquidity Provision in Limit Order Markets,'' \textit{Econometrica} \textbf{81}, 363--392.
\item Back, K., 2014, ``A Characterization of the Coskewness-Cokurtosis Pricing Model,'' \textit{Economics Letters} \textbf{125}, 219--222.  
\item Back, K., and K. Crotty, 2015, ``The Informational Role of Stock and Bond Volume,'' \textit{Review of Financial Studies} \textbf{28}, 1381--1427.
\item Back, K., Crotty, K., and T. Li, 2018, ``Identifying Information Asymmetry in Securities Markets,'' \textit{Review of Financial Studies} \textbf{31}, 2277--2325.
\item Back, K., Crane, A., and K. Crotty, 2018, ``Skewness Consequences of Seeking Alpha,'' \textit{Review of Financial Studies}, \textbf{31}, 4720--4761.
\item Back, K., Collin-Dufresne, P., Fos, V., Li, T., and A. Ljungqvist, 2018, ``Activism, Strategic Trading, and Liquidity,'' \textit{Econometrica} \textbf{86}, 1431--1463.
\item Back, K., Liu, R., and A. Teguia, 2019, ``Increasing Risk Aversion and Life-Cycle Investing,'' \textit{Mathematics and Financial Economics}, \textbf{13}, 287--302.
\item Back, K., Liu, R., and A. Teguia, 2020, ``Signaling in OTC Markets: Benefits and Costs of Transparency,'' \textit{Journal of Financial and Quantitative Analysis}, \textbf{55}, 47--75.
\item Back, K., and P. Barton, 2022, ``Mediation and Strategic Delay in Bargaining and Markets,'' \textit{Journal of Economic Dynamics and Control}, \textbf{141}.
\item Back, K., Crotty, K., and S. M. Kazempour, 2022, ``Validity, Tightness, and Forecasting Power of Risk Premium Bounds,''  \textit{Journal of Financial Economics}, \textbf{144}, 732--760.
    \item Back, K., Cocquemas, F., Ekren, I., and A. Lioui, 2021, ``Optimal Transport and Risk Aversion in Kyle's Model of Informed Trading,'' Under Revision.
\item Back, K., Carlin, B., Kazempour, S. M., and C. Xie, 2023, ``American Disclosure Options,'' Under Review.
\item Back, K., \c{C}elebi, O., Kakhbod, A., and A. M. Reppen, 2023, ``Segmented Trading Markets,'' Under Review.
\end{enumerate}

\end{document}

Teaching Statement

I have taught every topic in a finance curriculum at some point in my career.  Recently, I have been focused on helping students to expand their financial analysis toolkits beyond spreadsheets.  I discovered python roughly a dozen years ago and regard it as a beautiful language and an incredible tool.  It is a tool that I think our students will need in the future (see, for example, Microsoft's beta testing of Python in Excel) and should be learning now.  I teach students in a variety of programs: Ph.D. students in business and economics, MBA students, undergraduate business students, and students in the Masters of Data Science (MDS) program in the computer science department.  Outside of teaching asset pricing theory to the Ph.D. students in BUSI 521, using my book Asset Pricing and Portfolio Choice Theory (Oxford University Press, 2nd ed.), all of my other teaching involves the use of python to some extent.  This has been most successful with Ph.D. students (BUSI 520) and in the MDS courses (BUSI 721 and 722).  It has been less successful with MBA students.  The MBA students get valuable exposure (and exposure that they largely seem to value) but not enough exposure to actually acquire skills, given that almost all other courses in the business school are still spreadsheet-based.  I am optimistic that generative AI will bridge this gap.  I am pioneering a course in the second half of the spring semester for MBA students (MGMT 675: AI-Assisted Financial Analysis) that is aimed to develop prompt engineering skills to get ChatGPT to write python code for financial analyses.  I will cover some applications the students have seen before but with different methods and also some applications that are infeasible with spreadsheets.

Another major teaching-related emphasis of mine in the past few years has been the development of a website that illustrates investment concepts with the help of interactive figures and tables (https://learn-investments.rice-business.org).  I developed the website jointly with Kevin Crotty, an Associate Professor of Finance at JGSB.  The website contains over 80 pages, each of which illustrates a concept in investments.  Many of the pages pull data from online sources.  Kevin and I won the Financial Management Association Innovation in Teaching Award in 2023 for the website.  Python is not needed to use the website, but we do provide python code to conduct the analyses for students who want to learn it.

My contributions in teaching led to my being selectd recently by the Financial Management Association to be one of two nominees for the position of Vice President for Education.  The election will be in the spring.